%
% To familiarize yourself with this template, the body contains
% some examples of its use.  Look them over.  Then you can
% run LaTeX on this file.  After you have LaTeXed this file then
% you can look over the result either by printing it out with
% dvips or using xdvi. "pdflatex template.tex" should also work.
%

\documentclass[twoside]{article}
\setlength{\oddsidemargin}{0.25 in}
\setlength{\evensidemargin}{-0.25 in}
\setlength{\topmargin}{-0.6 in}
\setlength{\textwidth}{6.5 in}
\setlength{\textheight}{8.5 in}
\setlength{\headsep}{0.75 in}
\setlength{\parindent}{0 in}
\setlength{\parskip}{0.1 in}

%
% ADD PACKAGES here:
%

\usepackage{amsmath,amsfonts,graphicx,amssymb}

%
% The following commands set up the lecnum (lecture number)
% counter and make various numbering schemes work relative
% to the lecture number.
%
% \newcounter{lecnum}
% \renewcommand{\thepage}{\thelecnum-\arabic{page}}
% \renewcommand{\thesection}{\thelecnum.\arabic{section}}
% \renewcommand{\theequation}{\thelecnum.\arabic{equation}}
% \renewcommand{\thefigure}{\thelecnum.\arabic{figure}}
% \renewcommand{\thetable}{\thelecnum.\arabic{table}}

%
% The following macro is used to generate the header.
%
\newcommand{\lecture}[2]{
   \pagestyle{myheadings}
   \thispagestyle{plain}
   \newpage
%    \setcounter{lecnum}{#1}
   \setcounter{page}{1}
   \noindent
   \begin{center}
   \framebox{
      \vbox{\vspace{2mm}
    \hbox to 6.28in { {\bf CS698W: Game Theory and Collective Choice
	\hfill Jul-Nov 2017} }
       \vspace{4mm}
       \hbox to 6.28in { {\Large \hfill Project Report: #1  \hfill} }
       \vspace{2mm}
       \hbox to 6.28in { {\it Group member(s): #2 \hfill } }
      \vspace{2mm}}
   }
   \end{center}
   \markboth{Project: #1}{Project: #1}

%    This LaTeX template is by courtesy of UC Berkeley EECS dept.

%    {\bf Disclaimer}: {\it These notes aggregate content from several texts and have not been subjected to the usual scrutiny deserved by formal publications. If you find errors, please bring to the notice of the Instructor at swaprava.cse.iitk.ac.in.}
   \vspace*{1mm}
}
%
% Convention for citations is authors' initials followed by the year.
% For example, to cite a paper by Leighton and Maggs you would type
% \cite{LM89}, and to cite a paper by Strassen you would type \cite{S69}.
% (To avoid bibliography problems, for now we redefine the \cite command.)
% Also commands that create a suitable format for the reference list.
\renewcommand{\cite}[1]{[#1]}
\def\beginrefs{\begin{list}%
        {[\arabic{equation}]}{\usecounter{equation}
         \setlength{\leftmargin}{2.0truecm}\setlength{\labelsep}{0.4truecm}%
         \setlength{\labelwidth}{1.6truecm}}}
\def\endrefs{\end{list}}
\def\bibentry#1{\item[\hbox{[#1]}]}

%Use this command for a figure; it puts a figure in wherever you want it.
%usage: \fig{NUMBER}{JUST-THE-FILENAME}{CAPTION}
\newcommand{\fig}[3]{
% 			\vspace{#2}
			\begin{center}
			\includegraphics{figures/#2} 
			\newline
			Figure \thelecnum.#1:~#3
			\end{center}
	}
	
% Use these for theorems, lemmas, proofs, etc.
\newtheorem{theorem}{Theorem}%[lecnum]
\newtheorem{lemma}[theorem]{Lemma}
\newtheorem{proposition}[theorem]{Proposition}
\newtheorem{claim}[theorem]{Claim}
\newtheorem{corollary}[theorem]{Corollary}
\newtheorem{definition}[theorem]{Definition}
\newenvironment{proof}{{\bf Proof:}}{\hfill\rule{2mm}{2mm}}

% **** IF YOU WANT TO DEFINE ADDITIONAL MACROS FOR YOURSELF, PUT THEM HERE:

\newcommand\E{\mathbb{E}}
\DeclareMathOperator*{\argmin}{arg\,min}
\DeclareMathOperator*{\argmax}{arg\,max}

\begin{document}
%FILL IN THE RIGHT INFO.
%\lecture{**TITLE OF PROJECT**}{**TEAM MEMBERS**}
\lecture{That Awesome Project}{Name 1, Name 2, and Name 3}
%\footnotetext{These notes are partially based on those of Nigel Mansell.}

% **** YOUR NOTES GO HERE:

% Some general latex examples and examples making use of the
% macros follow.  
%**** IN GENERAL, BE BRIEF. LONG SCRIBE NOTES, NO MATTER HOW WELL WRITTEN,
%**** ARE NEVER READ BY ANYBODY.
% This lecture's notes illustrate some uses of
% various \LaTeX\ macros.  
% Take a look at this and imitate.

\begin{abstract}
 This is a one paragraph summary of your work. It should have the objectives, methods used (theoretical or experimental or both), and results. 
\end{abstract}


\section{Introduction to the problem}

This includes the motivation of the problem: What problem are you considering? Why is this problem important? 

\subsection{Related work}

A comprehensive and exhaustive survey of the literature. What part of the problem or related problem has already been solved? How different is your considered problem from that OR how does it complement that? For example: \cite{CW87} shows result A and this is improved in \cite{SLB08}. We pick a special subclass of this problem and provide a complete solution etc.

\subsection{Brief overview of the report}

This section explains which section does what. For example, we introduce the model is Section 2. Prove our main result in Section 3. Report our experimental findings in Section 4. We conclude this report in Section 5.

\section{Formal model of the problem}
{\bf [if it is an experimental project, this section should read as the ``formal setting of the experiments'' -- which should also have all the terms/notation of the experiments defined in this one place]}

This section sets up the notation and definitions you will be using in the rest of the report. Note that a better organized report should have all notation and definitions at one place so that someone navigating through the report knows where to find the definition of a term when found in some place in the paper. Only the definitions/explanatory terms that are very specific to a section, e.g., if a new abbreviation or term is introduced in the experiment section, should be introduced in that section.

\section{Main results/findings}

\begin{theorem}
\label{thm:first}
This is the first theorem.
\end{theorem}

\begin{proof}
This is the proof of the first theorem. If proofs are long, place that in appendix.
\end{proof}


\section{Experiments/Simulations}

Explain what is the setup of the experiments. For example, we randomly generate points on a unit square and create Erd\"os-Renyi graphs with connection probability $p$, where $p$ varies from $0.1$ to $0.9$ in steps of $0.1$ etc. If it is based on real datasets, mention where the datasets are available. The goal of this section should be to give enough information so that the reader can reproduce the experiment if need be.

\section{Summary and Discussions}

Summarize what has been done in different sections, what can we conclude from it, what is a good future direction of this work.

\section*{References}
\beginrefs
\bibentry{CW87}{\sc D.~Coppersmith} and {\sc S.~Winograd}, 
``Matrix multiplication via arithmetic progressions,''
{\it Proceedings of the 19th ACM Symposium on Theory of Computing},
1987, pp.~1--6.
\bibentry{SLB08}Shoham, Yoav, and Kevin Leyton-Brown. ``Multiagent systems: Algorithmic, game-theoretic, and logical foundations''. Cambridge University Press, 2008.
\endrefs

\section*{Important Information}

{\bf Weight Distribution:} 10\% for the motivation of the problem -- Section 1, 30\% on the literature survey -- Section 1.1, 40\% for the main (theoretical + experimental) results -- Sections 3 and 4 together -- this weightage cannot be further broken into sections, since different projects may have dissimilar amount of theory and experimental content, 20\% for the presentation.

{\bf Page Limit:} 5 pages including the references in this format. Do not change the font sizes or dimensions of the template. This limit does not count the appendices -- which can have any length. Note that appendices are to support the claims in the main sections -- and therefore that does not directly count towards the evaluation.

{\bf Note on presentation:} The presentation should be ideally for 15 minutes. This should roughly follow the pattern of the report, but explaining the intuition and broad results rather than the details.

%% USE THE FOLLOWING PART ONLY IF YOU HAVE AN APPENDIX
\begin{center}
 \Large{\bf Appendices}
\end{center}

\appendix

\section{Proof of Theorem~\ref{thm:first}}

Take your space to give the full details of the proof. Appendices do not count to the page limit.

\section{Additional Experiments}

Plots that could not be presented in Section 4.

\end{document}




